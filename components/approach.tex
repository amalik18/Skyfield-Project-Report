\documentclass[../main.tex]{subfiles}
\usepackage[utf8x]{inputenc}
\usepackage{blindtext}
\usepackage{float}
\usepackage{graphicx}
\usepackage{siunitx}

\begin{document}

We approached this project as a purely programmatic one, the capability to gather the necessary data was there, it just needed to be leveraged properly. The approach included it a complete Pythonic command-line interface (CLI), which provides the user with an easy-to-use method to input critical data to determine the azimuth, elevation, and distance for a given celestial body or satellite, all referenced from a specific location on Earth at a provided time. Skyfield can ingest ephemeris data as either an SPK file or a TLE list for an Earth satellite. Once the data is provided, this library can then be used to determine the azimuth and altitude for celestial bodies and satellites. In this specific project, the library was used to determine the azimuth and altitude for the Moon, Mars, the Voyager 1 probe, and the ISS ZARYA from the Nguyen Engineering Building at George Mason University. The values generated by the Skyfield API were then verified against the Sky Live Online Planetarium which confirmed the validity of the output. This code was also run from an embedded environment, running on a Raspberry Pi 4B.


\end{document}