\documentclass[../main.tex]{subfiles}
\usepackage[utf8x]{inputenc}
\usepackage{blindtext}
\usepackage{float}
\usepackage{graphicx}
\usepackage{siunitx}

\begin{document}

The purpose of this project was to evaluate the Skyfield Python library to create a proof-of-concept/rudimentary implementation to aid in antenna tracking for George Mason University's 9M Dish. Skyfield is an open source Python library which "computes positions for the stars, planets, and satellites in orbit around the Earth. Its results should agree with the positions generated by the United States Naval Observatory and their Astronomical Almanac to within 0.0005 arcseconds" \cite{skyfield-main}. Skyfield, mainly, works using SPK files, which define the ephemeris of orbiting bodies\cite{SPICE-concept}. NASA provides SPK files for most celestial bodied and some man-made satellites. Skyfield also integrates with the \textit{wgs84} standard, which allows topocentric location usage. This allows calculation of position differences from specific locations on the Earth's surface to another orbiting body, and allows for two-line element (TLE) data integration\cite{skyfield-main}. This project focused on generating antenna pointing information, initially, for three objects, the Moon, Mars, and the Voyager 1 space probe. 

\end{document}